\documentclass[ignorenonframetext,]{beamer}
\setbeamertemplate{caption}[numbered]
\setbeamertemplate{caption label separator}{: }
\setbeamercolor{caption name}{fg=normal text.fg}
\beamertemplatenavigationsymbolsempty
\usepackage{lmodern}
\usepackage{amssymb,amsmath}
\usepackage{ifxetex,ifluatex}
\usepackage{fixltx2e} % provides \textsubscript
\ifnum 0\ifxetex 1\fi\ifluatex 1\fi=0 % if pdftex
  \usepackage[T1]{fontenc}
  \usepackage[utf8]{inputenc}
\else % if luatex or xelatex
  \ifxetex
    \usepackage{mathspec}
  \else
    \usepackage{fontspec}
  \fi
  \defaultfontfeatures{Ligatures=TeX,Scale=MatchLowercase}
    \setmainfont[]{IPAPMincho}
\fi
\usefonttheme{serif} % use mainfont rather than sansfont for slide text
% use upquote if available, for straight quotes in verbatim environments
\IfFileExists{upquote.sty}{\usepackage{upquote}}{}
% use microtype if available
\IfFileExists{microtype.sty}{%
\usepackage{microtype}
\UseMicrotypeSet[protrusion]{basicmath} % disable protrusion for tt fonts
}{}
\newif\ifbibliography
\hypersetup{
            pdftitle={Iteration - r4ds},
            pdfauthor={Tomoya Fukumoto},
            pdfborder={0 0 0},
            breaklinks=true}
\urlstyle{same}  % don't use monospace font for urls
\usepackage{color}
\usepackage{fancyvrb}
\newcommand{\VerbBar}{|}
\newcommand{\VERB}{\Verb[commandchars=\\\{\}]}
\DefineVerbatimEnvironment{Highlighting}{Verbatim}{commandchars=\\\{\}}
% Add ',fontsize=\small' for more characters per line
\usepackage{framed}
\definecolor{shadecolor}{RGB}{248,248,248}
\newenvironment{Shaded}{\begin{snugshade}}{\end{snugshade}}
\newcommand{\KeywordTok}[1]{\textcolor[rgb]{0.13,0.29,0.53}{\textbf{#1}}}
\newcommand{\DataTypeTok}[1]{\textcolor[rgb]{0.13,0.29,0.53}{#1}}
\newcommand{\DecValTok}[1]{\textcolor[rgb]{0.00,0.00,0.81}{#1}}
\newcommand{\BaseNTok}[1]{\textcolor[rgb]{0.00,0.00,0.81}{#1}}
\newcommand{\FloatTok}[1]{\textcolor[rgb]{0.00,0.00,0.81}{#1}}
\newcommand{\ConstantTok}[1]{\textcolor[rgb]{0.00,0.00,0.00}{#1}}
\newcommand{\CharTok}[1]{\textcolor[rgb]{0.31,0.60,0.02}{#1}}
\newcommand{\SpecialCharTok}[1]{\textcolor[rgb]{0.00,0.00,0.00}{#1}}
\newcommand{\StringTok}[1]{\textcolor[rgb]{0.31,0.60,0.02}{#1}}
\newcommand{\VerbatimStringTok}[1]{\textcolor[rgb]{0.31,0.60,0.02}{#1}}
\newcommand{\SpecialStringTok}[1]{\textcolor[rgb]{0.31,0.60,0.02}{#1}}
\newcommand{\ImportTok}[1]{#1}
\newcommand{\CommentTok}[1]{\textcolor[rgb]{0.56,0.35,0.01}{\textit{#1}}}
\newcommand{\DocumentationTok}[1]{\textcolor[rgb]{0.56,0.35,0.01}{\textbf{\textit{#1}}}}
\newcommand{\AnnotationTok}[1]{\textcolor[rgb]{0.56,0.35,0.01}{\textbf{\textit{#1}}}}
\newcommand{\CommentVarTok}[1]{\textcolor[rgb]{0.56,0.35,0.01}{\textbf{\textit{#1}}}}
\newcommand{\OtherTok}[1]{\textcolor[rgb]{0.56,0.35,0.01}{#1}}
\newcommand{\FunctionTok}[1]{\textcolor[rgb]{0.00,0.00,0.00}{#1}}
\newcommand{\VariableTok}[1]{\textcolor[rgb]{0.00,0.00,0.00}{#1}}
\newcommand{\ControlFlowTok}[1]{\textcolor[rgb]{0.13,0.29,0.53}{\textbf{#1}}}
\newcommand{\OperatorTok}[1]{\textcolor[rgb]{0.81,0.36,0.00}{\textbf{#1}}}
\newcommand{\BuiltInTok}[1]{#1}
\newcommand{\ExtensionTok}[1]{#1}
\newcommand{\PreprocessorTok}[1]{\textcolor[rgb]{0.56,0.35,0.01}{\textit{#1}}}
\newcommand{\AttributeTok}[1]{\textcolor[rgb]{0.77,0.63,0.00}{#1}}
\newcommand{\RegionMarkerTok}[1]{#1}
\newcommand{\InformationTok}[1]{\textcolor[rgb]{0.56,0.35,0.01}{\textbf{\textit{#1}}}}
\newcommand{\WarningTok}[1]{\textcolor[rgb]{0.56,0.35,0.01}{\textbf{\textit{#1}}}}
\newcommand{\AlertTok}[1]{\textcolor[rgb]{0.94,0.16,0.16}{#1}}
\newcommand{\ErrorTok}[1]{\textcolor[rgb]{0.64,0.00,0.00}{\textbf{#1}}}
\newcommand{\NormalTok}[1]{#1}

% Prevent slide breaks in the middle of a paragraph:
\widowpenalties 1 10000
\raggedbottom

\AtBeginPart{
  \let\insertpartnumber\relax
  \let\partname\relax
  \frame{\partpage}
}
\AtBeginSection{
  \ifbibliography
  \else
    \let\insertsectionnumber\relax
    \let\sectionname\relax
    \frame{\sectionpage}
  \fi
}
\AtBeginSubsection{
  \let\insertsubsectionnumber\relax
  \let\subsectionname\relax
  \frame{\subsectionpage}
}

\setlength{\parindent}{0pt}
\setlength{\parskip}{6pt plus 2pt minus 1pt}
\setlength{\emergencystretch}{3em}  % prevent overfull lines
\providecommand{\tightlist}{%
  \setlength{\itemsep}{0pt}\setlength{\parskip}{0pt}}
\setcounter{secnumdepth}{0}
\usepackage{zxjatype}
\usepackage[ipa]{zxjafont}

\title{Iteration - r4ds}
\author{Tomoya Fukumoto}
\date{2019-09-06}

\begin{document}
\frame{\titlepage}

\begin{frame}{Iteration}

繰り返し作業をどうやって自動化するための二つの手法

\begin{enumerate}
\def\labelenumi{\arabic{enumi}.}
\tightlist
\item
  ループ
\item
  関数型プログラミング(functional programming)
\end{enumerate}

\end{frame}

\begin{frame}[fragile]{準備}

\begin{Shaded}
\begin{Highlighting}[]
\KeywordTok{library}\NormalTok{(tidyverse)}
\end{Highlighting}
\end{Shaded}

ループに関わるのは\texttt{base}ライブラリ

FPに関わるのは\texttt{purrr}ライブラリ

\end{frame}

\begin{frame}{21.2 For loops}

最も標準的なループ

\end{frame}

\begin{frame}[fragile]{例:各行のmedianを求める(ループなし)}

\begin{Shaded}
\begin{Highlighting}[]
\NormalTok{df <-}\StringTok{ }\KeywordTok{tibble}\NormalTok{(}
         \DataTypeTok{a =} \KeywordTok{rnorm}\NormalTok{(}\DecValTok{10}\NormalTok{),}
         \DataTypeTok{b =} \KeywordTok{rnorm}\NormalTok{(}\DecValTok{10}\NormalTok{),}
         \DataTypeTok{c =} \KeywordTok{rnorm}\NormalTok{(}\DecValTok{10}\NormalTok{),}
         \DataTypeTok{d =} \KeywordTok{rnorm}\NormalTok{(}\DecValTok{10}\NormalTok{)}
\NormalTok{) }
\KeywordTok{median}\NormalTok{(df}\OperatorTok{$}\NormalTok{a)}
\KeywordTok{median}\NormalTok{(df}\OperatorTok{$}\NormalTok{b)}
\KeywordTok{median}\NormalTok{(df}\OperatorTok{$}\NormalTok{c)}
\KeywordTok{median}\NormalTok{(df}\OperatorTok{$}\NormalTok{d)}
\end{Highlighting}
\end{Shaded}

\end{frame}

\begin{frame}[fragile]{例:各行のmedianを求める(ループ)}

\begin{Shaded}
\begin{Highlighting}[]
\NormalTok{output <-}\StringTok{ }\KeywordTok{vector}\NormalTok{(}\StringTok{"double"}\NormalTok{, }\KeywordTok{ncol}\NormalTok{(df))  }\CommentTok{# 1. output}
\ControlFlowTok{for}\NormalTok{ (i }\ControlFlowTok{in} \KeywordTok{seq_along}\NormalTok{(df)) \{            }\CommentTok{# 2. sequence}
\NormalTok{  output[[i]] <-}\StringTok{ }\KeywordTok{median}\NormalTok{(df[[i]])      }\CommentTok{# 3. body}
\NormalTok{\}}
\NormalTok{output}
\end{Highlighting}
\end{Shaded}

\begin{verbatim}
## [1] -0.23397728 -0.35668072  0.01067616 -0.31417659
\end{verbatim}

\end{frame}

\begin{frame}[fragile]{ループの構成要素 output}

\begin{Shaded}
\begin{Highlighting}[]
\NormalTok{output <-}\StringTok{ }\KeywordTok{vector}\NormalTok{(}\StringTok{"double"}\NormalTok{, }\KeywordTok{ncol}\NormalTok{(df))}
\end{Highlighting}
\end{Shaded}

\begin{itemize}
\tightlist
\item
  ループの出力の器
\item
  ループが始まる前に作る
\item
  \texttt{vector}関数で型と長さを指定する
\item
  長さを指定せずとも処理が遅くなる
\end{itemize}

\end{frame}

\begin{frame}[fragile]{ループの構成要素 sequence}

\begin{Shaded}
\begin{Highlighting}[]
\NormalTok{i }\ControlFlowTok{in} \KeywordTok{seq_along}\NormalTok{(df)}
\end{Highlighting}
\end{Shaded}

\begin{itemize}
\tightlist
\item
  どうループを回すか
\item
  一周するたびに\texttt{i}がベクトル\texttt{seq\_along(df)}の中で値を変化させる
\item
  (\texttt{seq\_along(df)}は\texttt{1:length(df)}と\textbf{ほぼ}同じ)
\end{itemize}

\end{frame}

\begin{frame}[fragile]{ループの構成要素 body}

\begin{Shaded}
\begin{Highlighting}[]
\NormalTok{output[[i]] <-}\StringTok{ }\KeywordTok{median}\NormalTok{(df[[i]])}
\end{Highlighting}
\end{Shaded}

\begin{itemize}
\tightlist
\item
  ループで実際に処理する内容
\item
  1回目は\texttt{output{[}{[}1{]}{]}\ \textless{}-\ median(df{[}{[}1{]}{]})}
\item
  2回目は\texttt{output{[}{[}2{]}{]}\ \textless{}-\ median(df{[}{[}2{]}{]})}
\end{itemize}

\end{frame}

\begin{frame}{参考文献}

\end{frame}

\end{document}
