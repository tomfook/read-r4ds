\documentclass[ignorenonframetext,]{beamer}
\setbeamertemplate{caption}[numbered]
\setbeamertemplate{caption label separator}{: }
\setbeamercolor{caption name}{fg=normal text.fg}
\beamertemplatenavigationsymbolsempty
\usepackage{lmodern}
\usepackage{amssymb,amsmath}
\usepackage{ifxetex,ifluatex}
\usepackage{fixltx2e} % provides \textsubscript
\ifnum 0\ifxetex 1\fi\ifluatex 1\fi=0 % if pdftex
  \usepackage[T1]{fontenc}
  \usepackage[utf8]{inputenc}
\else % if luatex or xelatex
  \ifxetex
    \usepackage{mathspec}
  \else
    \usepackage{fontspec}
  \fi
  \defaultfontfeatures{Ligatures=TeX,Scale=MatchLowercase}
    \setmainfont[]{IPAMincho}
\fi
\usefonttheme{serif} % use mainfont rather than sansfont for slide text
% use upquote if available, for straight quotes in verbatim environments
\IfFileExists{upquote.sty}{\usepackage{upquote}}{}
% use microtype if available
\IfFileExists{microtype.sty}{%
\usepackage{microtype}
\UseMicrotypeSet[protrusion]{basicmath} % disable protrusion for tt fonts
}{}
\newif\ifbibliography
\hypersetup{
            pdftitle={Functions - r4ds},
            pdfauthor={Tomoya Fukumoto},
            pdfborder={0 0 0},
            breaklinks=true}
\urlstyle{same}  % don't use monospace font for urls
\usepackage{color}
\usepackage{fancyvrb}
\newcommand{\VerbBar}{|}
\newcommand{\VERB}{\Verb[commandchars=\\\{\}]}
\DefineVerbatimEnvironment{Highlighting}{Verbatim}{commandchars=\\\{\}}
% Add ',fontsize=\small' for more characters per line
\usepackage{framed}
\definecolor{shadecolor}{RGB}{248,248,248}
\newenvironment{Shaded}{\begin{snugshade}}{\end{snugshade}}
\newcommand{\KeywordTok}[1]{\textcolor[rgb]{0.13,0.29,0.53}{\textbf{#1}}}
\newcommand{\DataTypeTok}[1]{\textcolor[rgb]{0.13,0.29,0.53}{#1}}
\newcommand{\DecValTok}[1]{\textcolor[rgb]{0.00,0.00,0.81}{#1}}
\newcommand{\BaseNTok}[1]{\textcolor[rgb]{0.00,0.00,0.81}{#1}}
\newcommand{\FloatTok}[1]{\textcolor[rgb]{0.00,0.00,0.81}{#1}}
\newcommand{\ConstantTok}[1]{\textcolor[rgb]{0.00,0.00,0.00}{#1}}
\newcommand{\CharTok}[1]{\textcolor[rgb]{0.31,0.60,0.02}{#1}}
\newcommand{\SpecialCharTok}[1]{\textcolor[rgb]{0.00,0.00,0.00}{#1}}
\newcommand{\StringTok}[1]{\textcolor[rgb]{0.31,0.60,0.02}{#1}}
\newcommand{\VerbatimStringTok}[1]{\textcolor[rgb]{0.31,0.60,0.02}{#1}}
\newcommand{\SpecialStringTok}[1]{\textcolor[rgb]{0.31,0.60,0.02}{#1}}
\newcommand{\ImportTok}[1]{#1}
\newcommand{\CommentTok}[1]{\textcolor[rgb]{0.56,0.35,0.01}{\textit{#1}}}
\newcommand{\DocumentationTok}[1]{\textcolor[rgb]{0.56,0.35,0.01}{\textbf{\textit{#1}}}}
\newcommand{\AnnotationTok}[1]{\textcolor[rgb]{0.56,0.35,0.01}{\textbf{\textit{#1}}}}
\newcommand{\CommentVarTok}[1]{\textcolor[rgb]{0.56,0.35,0.01}{\textbf{\textit{#1}}}}
\newcommand{\OtherTok}[1]{\textcolor[rgb]{0.56,0.35,0.01}{#1}}
\newcommand{\FunctionTok}[1]{\textcolor[rgb]{0.00,0.00,0.00}{#1}}
\newcommand{\VariableTok}[1]{\textcolor[rgb]{0.00,0.00,0.00}{#1}}
\newcommand{\ControlFlowTok}[1]{\textcolor[rgb]{0.13,0.29,0.53}{\textbf{#1}}}
\newcommand{\OperatorTok}[1]{\textcolor[rgb]{0.81,0.36,0.00}{\textbf{#1}}}
\newcommand{\BuiltInTok}[1]{#1}
\newcommand{\ExtensionTok}[1]{#1}
\newcommand{\PreprocessorTok}[1]{\textcolor[rgb]{0.56,0.35,0.01}{\textit{#1}}}
\newcommand{\AttributeTok}[1]{\textcolor[rgb]{0.77,0.63,0.00}{#1}}
\newcommand{\RegionMarkerTok}[1]{#1}
\newcommand{\InformationTok}[1]{\textcolor[rgb]{0.56,0.35,0.01}{\textbf{\textit{#1}}}}
\newcommand{\WarningTok}[1]{\textcolor[rgb]{0.56,0.35,0.01}{\textbf{\textit{#1}}}}
\newcommand{\AlertTok}[1]{\textcolor[rgb]{0.94,0.16,0.16}{#1}}
\newcommand{\ErrorTok}[1]{\textcolor[rgb]{0.64,0.00,0.00}{\textbf{#1}}}
\newcommand{\NormalTok}[1]{#1}

% Prevent slide breaks in the middle of a paragraph:
\widowpenalties 1 10000
\raggedbottom

\AtBeginPart{
  \let\insertpartnumber\relax
  \let\partname\relax
  \frame{\partpage}
}
\AtBeginSection{
  \ifbibliography
  \else
    \let\insertsectionnumber\relax
    \let\sectionname\relax
    \frame{\sectionpage}
  \fi
}
\AtBeginSubsection{
  \let\insertsubsectionnumber\relax
  \let\subsectionname\relax
  \frame{\subsectionpage}
}

\setlength{\parindent}{0pt}
\setlength{\parskip}{6pt plus 2pt minus 1pt}
\setlength{\emergencystretch}{3em}  % prevent overfull lines
\providecommand{\tightlist}{%
  \setlength{\itemsep}{0pt}\setlength{\parskip}{0pt}}
\setcounter{secnumdepth}{0}
\usepackage{zxjatype}
\usepackage[ipa]{zxjafont}

\title{Functions - r4ds}
\author{Tomoya Fukumoto}
\date{2019-07-26}

\begin{document}
\frame{\titlepage}

\begin{frame}{関数}

\begin{block}{神のお言葉}

\begin{quote}
データサイエンティストとしてレベルアップする最高の方法は関数を書くこと
\end{quote}

\end{block}

\begin{block}{関数の利点(コピペに対する)}

\begin{enumerate}
\def\labelenumi{\arabic{enumi}.}
\tightlist
\item
  ある処理の塊に名前を付けて管理できる。

  \begin{itemize}
  \tightlist
  \item
    コードが理解しやすくなる
  \end{itemize}
\item
  変更があったときに一箇所だけ修正すればよい

  \begin{itemize}
  \tightlist
  \item
    生産性UP!
  \end{itemize}
\item
  コピペしたときのミスの可能性を減らせる

  \begin{itemize}
  \tightlist
  \item
    不具合の減少
  \end{itemize}
\end{enumerate}

\end{block}

\end{frame}

\begin{frame}{19.2 When you should you write a function?}

どういうときに関数を書くのか?

A. コピペの回数が2回を上回るとき

{ Don't Repeat Yourself (DRY) principle}

\end{frame}

\begin{frame}[fragile]{関数を書くべき例}

\begin{Shaded}
\begin{Highlighting}[]
\NormalTok{df <-}\StringTok{ }\NormalTok{tibble}\OperatorTok{::}\KeywordTok{tibble}\NormalTok{(}
             \DataTypeTok{a =} \KeywordTok{rnorm}\NormalTok{(}\DecValTok{10}\NormalTok{),}
             \DataTypeTok{b =} \KeywordTok{rnorm}\NormalTok{(}\DecValTok{10}\NormalTok{), }
             \DataTypeTok{c =} \KeywordTok{rnorm}\NormalTok{(}\DecValTok{10}\NormalTok{),}
             \DataTypeTok{d =} \KeywordTok{rnorm}\NormalTok{(}\DecValTok{10}\NormalTok{))}

\NormalTok{df}\OperatorTok{$}\NormalTok{a <-}\StringTok{ }\NormalTok{(df}\OperatorTok{$}\NormalTok{a }\OperatorTok{-}\StringTok{ }\KeywordTok{min}\NormalTok{(df}\OperatorTok{$}\NormalTok{a, }\DataTypeTok{na.rm =} \OtherTok{TRUE}\NormalTok{)) }\OperatorTok{/}
\StringTok{  }\NormalTok{(}\KeywordTok{max}\NormalTok{(df}\OperatorTok{$}\NormalTok{a, }\DataTypeTok{na.rm =} \OtherTok{TRUE}\NormalTok{) }\OperatorTok{-}\StringTok{ }\KeywordTok{min}\NormalTok{(df}\OperatorTok{$}\NormalTok{a, }\DataTypeTok{na.rm =} \OtherTok{TRUE}\NormalTok{))}
\NormalTok{df}\OperatorTok{$}\NormalTok{b <-}\StringTok{ }\NormalTok{(df}\OperatorTok{$}\NormalTok{b }\OperatorTok{-}\StringTok{ }\KeywordTok{min}\NormalTok{(df}\OperatorTok{$}\NormalTok{b, }\DataTypeTok{na.rm =} \OtherTok{TRUE}\NormalTok{)) }\OperatorTok{/}
\StringTok{  }\NormalTok{(}\KeywordTok{max}\NormalTok{(df}\OperatorTok{$}\NormalTok{b, }\DataTypeTok{na.rm =} \OtherTok{TRUE}\NormalTok{) }\OperatorTok{-}\StringTok{ }\KeywordTok{min}\NormalTok{(df}\OperatorTok{$}\NormalTok{a, }\DataTypeTok{na.rm =} \OtherTok{TRUE}\NormalTok{))}
\NormalTok{df}\OperatorTok{$}\NormalTok{c <-}\StringTok{ }\NormalTok{(df}\OperatorTok{$}\NormalTok{c }\OperatorTok{-}\StringTok{ }\KeywordTok{min}\NormalTok{(df}\OperatorTok{$}\NormalTok{c, }\DataTypeTok{na.rm =} \OtherTok{TRUE}\NormalTok{)) }\OperatorTok{/}
\StringTok{  }\NormalTok{(}\KeywordTok{max}\NormalTok{(df}\OperatorTok{$}\NormalTok{c, }\DataTypeTok{na.rm =} \OtherTok{TRUE}\NormalTok{) }\OperatorTok{-}\StringTok{ }\KeywordTok{min}\NormalTok{(df}\OperatorTok{$}\NormalTok{c, }\DataTypeTok{na.rm =} \OtherTok{TRUE}\NormalTok{))}
\NormalTok{df}\OperatorTok{$}\NormalTok{d <-}\StringTok{ }\NormalTok{(df}\OperatorTok{$}\NormalTok{d }\OperatorTok{-}\StringTok{ }\KeywordTok{min}\NormalTok{(df}\OperatorTok{$}\NormalTok{d, }\DataTypeTok{na.rm =} \OtherTok{TRUE}\NormalTok{)) }\OperatorTok{/}
\StringTok{  }\NormalTok{(}\KeywordTok{max}\NormalTok{(df}\OperatorTok{$}\NormalTok{d, }\DataTypeTok{na.rm =} \OtherTok{TRUE}\NormalTok{) }\OperatorTok{-}\StringTok{ }\KeywordTok{min}\NormalTok{(df}\OperatorTok{$}\NormalTok{d, }\DataTypeTok{na.rm =} \OtherTok{TRUE}\NormalTok{))}
\end{Highlighting}
\end{Shaded}

\end{frame}

\begin{frame}[fragile]{関数を書くためのstep1 コードを分析する}

処理の入力は?

\begin{Shaded}
\begin{Highlighting}[]
\NormalTok{(df}\OperatorTok{$}\NormalTok{a }\OperatorTok{-}\StringTok{ }\KeywordTok{min}\NormalTok{(df}\OperatorTok{$}\NormalTok{a, }\DataTypeTok{na.rm =} \OtherTok{TRUE}\NormalTok{)) }\OperatorTok{/}
\StringTok{    }\NormalTok{(}\KeywordTok{max}\NormalTok{(df}\OperatorTok{$}\NormalTok{a, }\DataTypeTok{na.rm =} \OtherTok{TRUE}\NormalTok{) }\OperatorTok{-}\StringTok{ }\KeywordTok{min}\NormalTok{(df}\OperatorTok{$}\NormalTok{a, }\DataTypeTok{na.rm =} \OtherTok{TRUE}\NormalTok{))}
\end{Highlighting}
\end{Shaded}

\begin{block}{答え}

\begin{Shaded}
\begin{Highlighting}[]
\NormalTok{x <-}\StringTok{ }\NormalTok{df}\OperatorTok{$}\NormalTok{a}
\NormalTok{(x }\OperatorTok{-}\StringTok{ }\KeywordTok{min}\NormalTok{(x, }\DataTypeTok{na.rm =} \OtherTok{TRUE}\NormalTok{)) }\OperatorTok{/}
\StringTok{  }\NormalTok{(}\KeywordTok{max}\NormalTok{(x, }\DataTypeTok{na.rm =} \OtherTok{TRUE}\NormalTok{) }\OperatorTok{-}\StringTok{ }\KeywordTok{min}\NormalTok{(x, }\DataTypeTok{na.rm =} \OtherTok{TRUE}\NormalTok{))}
\end{Highlighting}
\end{Shaded}

\end{block}

\end{frame}

\begin{frame}[fragile]{関数を作る}

\begin{Shaded}
\begin{Highlighting}[]
\NormalTok{rescale01 <-}\StringTok{ }\ControlFlowTok{function}\NormalTok{(x) \{}
\NormalTok{  rng <-}\StringTok{ }\KeywordTok{range}\NormalTok{(x, }\DataTypeTok{na.rm =} \OtherTok{TRUE}\NormalTok{)}
\NormalTok{  (x }\OperatorTok{-}\StringTok{ }\NormalTok{rng[}\DecValTok{1}\NormalTok{]) }\OperatorTok{/}\StringTok{ }\NormalTok{(rng[}\DecValTok{2}\NormalTok{] }\OperatorTok{-}\StringTok{ }\NormalTok{rng[}\DecValTok{1}\NormalTok{])}
\NormalTok{\}}
\end{Highlighting}
\end{Shaded}

\begin{block}{方法}

\begin{enumerate}
\def\labelenumi{\arabic{enumi}.}
\tightlist
\item
  関数の名前を決定する

  \begin{itemize}
  \tightlist
  \item
    \texttt{rescale01}
  \end{itemize}
\item
  入力、または引数を\texttt{function}の中に入れる
\item
  関数の内容を\texttt{function(...)}の後に続く\texttt{\{}のブロックで表現
\end{enumerate}

\end{block}

\end{frame}

\begin{frame}[fragile]{関数を使ってコードを書き直す}

\begin{Shaded}
\begin{Highlighting}[]
\NormalTok{df}\OperatorTok{$}\NormalTok{a <-}\StringTok{ }\KeywordTok{rescale01}\NormalTok{(df}\OperatorTok{$}\NormalTok{a)}
\NormalTok{df}\OperatorTok{$}\NormalTok{b <-}\StringTok{ }\KeywordTok{rescale01}\NormalTok{(df}\OperatorTok{$}\NormalTok{b)}
\NormalTok{df}\OperatorTok{$}\NormalTok{c <-}\StringTok{ }\KeywordTok{rescale01}\NormalTok{(df}\OperatorTok{$}\NormalTok{c)}
\NormalTok{df}\OperatorTok{$}\NormalTok{d <-}\StringTok{ }\KeywordTok{rescale01}\NormalTok{(df}\OperatorTok{$}\NormalTok{d)}
\end{Highlighting}
\end{Shaded}

すっきりした!

\begin{block}{まだコピペが残ってるやん}

\(\Rightarrow\) 次の次の章iterationを待て

\end{block}

\end{frame}

\begin{frame}{19.3 Functions are for humans and computers}

関数はコンピュータのためだけではなく人間のために書く

\begin{itemize}
\tightlist
\item
  関数名
\item
  コメント
\end{itemize}

\end{frame}

\begin{frame}{関数名の決め方}

\begin{itemize}
\tightlist
\item
  短く、しかし処理の内容を明確に表現したい

  \begin{itemize}
  \tightlist
  \item
    トレードオフになることも多い
  \item
    どちらか一方を選択するとすれば明確にする

    \begin{itemize}
    \tightlist
    \item
      RStudioのオートコンプリート機能
    \end{itemize}
  \end{itemize}
\item
  \textbf{名詞}ではなく\textbf{動詞}で

  \begin{itemize}
  \tightlist
  \item
    引数は名詞
  \item
    動詞が``get''や``compute''ならその目的語(名詞)もあり
  \end{itemize}
\item
  よりよい名前が見つかったら変更することをためらうな
\item
  関数群を作るときは前半部を統一する

  \begin{itemize}
  \tightlist
  \item
    オートコンプリート
  \end{itemize}
\end{itemize}

\end{frame}

\begin{frame}[fragile]{関数名の例}

\begin{Shaded}
\begin{Highlighting}[]
\CommentTok{# Too short}
\KeywordTok{f}\NormalTok{()}

\CommentTok{# Not a verb, or descriptive}
\KeywordTok{my_awesome_function}\NormalTok{()}

\CommentTok{# Long, but clear}
\KeywordTok{impute_missing}\NormalTok{()}
\KeywordTok{collapse_years}\NormalTok{()}
\end{Highlighting}
\end{Shaded}

\end{frame}

\begin{frame}[fragile]{関数群の命名}

\begin{Shaded}
\begin{Highlighting}[]
\CommentTok{# Good}
\KeywordTok{input_select}\NormalTok{()}
\KeywordTok{input_checkbox}\NormalTok{()}
\KeywordTok{input_text}\NormalTok{()}

\CommentTok{# Not so good}
\KeywordTok{select_input}\NormalTok{()}
\KeywordTok{checkbox_input}\NormalTok{()}
\KeywordTok{text_input}\NormalTok{() }
\end{Highlighting}
\end{Shaded}

\end{frame}

\begin{frame}[fragile]{命名規則}

\begin{itemize}
\tightlist
\item
  \texttt{snake\_case}

  \begin{itemize}
  \tightlist
  \item
    Hadleyおすすめ
  \end{itemize}
\item
  \texttt{camelCase}
\end{itemize}

どっちでもいいけど、どちらかに統一するべき

\end{frame}

\begin{frame}[fragile]{コメント}

\texttt{\#}でその行の\texttt{\#}より右側はコメントアウト

\begin{Shaded}
\begin{Highlighting}[]
\CommentTok{# this is comment}
\NormalTok{a <-}\StringTok{ }\NormalTok{pi }\CommentTok{#this also comment}
\end{Highlighting}
\end{Shaded}

\begin{itemize}
\tightlist
\item
  コメントは「なぜ」その処理をするのかを書くべし

  \begin{itemize}
  \tightlist
  \item
    「何を」「どのようにして」では\textbf{無い}
  \end{itemize}
\item
  なぜ

  \begin{itemize}
  \tightlist
  \item
    中間変数を置いたのか?
  \item
    2つの関数に分けたのか?
  \item
    他の方法は試したけどうまくいかなかったのか?
  \end{itemize}
\item
  なぜこのようなコードになったのか?
\end{itemize}

\end{frame}

\begin{frame}[fragile]{コードセクション(Rstudio)}

次のコードを使えばRStdudioでセクション区切りとみなされる

\begin{Shaded}
\begin{Highlighting}[]
\CommentTok{# Load data --------------------------------------}

\CommentTok{# Plot data --------------------------------------}
\end{Highlighting}
\end{Shaded}

\href{../img/rstudio-nav.png}{nav}

\begin{block}{キーボード・ショートカット}

Ctrl + Shift + R

\end{block}

\end{frame}

\end{document}
